
% This LaTeX was auto-generated from MATLAB code.
% To make changes, update the MATLAB code and republish this document.

\documentclass{article}
\usepackage{graphicx}
\usepackage{color}

\sloppy
\definecolor{lightgray}{gray}{0.5}
\setlength{\parindent}{0pt}

\begin{document}

    
    
\section*{}

\begin{par}

Decision Making - ELECTRE \\
Type P.$\alpha$ Problem.

\end{par} \vspace{1em}
\begin{par}
Problem dataset.
\end{par} \vspace{1em}
\begin{verbatim}
ProblemTable = 'ProblemTable.dat';
AG = readtable(ProblemTable);
sizeAG = size(AG) ;
disp(sizeAG(2))
save('ProblemTable.mat','AG')
load ProblemTable
\end{verbatim}

        \color{lightgray} \begin{verbatim}     4

\end{verbatim} \color{black}
    \begin{par}
Assessment dataset.
\end{par} \vspace{1em}
\begin{verbatim}
EvalTable = 'EvalTable.dat';
W = readtable(EvalTable);
save('EvalTable.mat','W')
load EvalTable
\end{verbatim}
\begin{par}
Thresholds values.
\end{par} \vspace{1em}
\begin{verbatim}
Thresholds = 'thresholds.dat';
T = readtable(Thresholds);
\end{verbatim}
\begin{par}
Normalization of weights
\end{par} \vspace{1em}
\begin{verbatim}
wsum=0;

for id = 1: size(W,1)
    wsum= wsum + W.Weight(id);
end
disp(wsum);
disp(W)
WW=W;
% Redo weights matrix with normalized weight values.
WW.Weight=(1/wsum)*W.Weight;
sizeWW = size(WW);
disp(WW)
\end{verbatim}

        \color{lightgray} \begin{verbatim}     9

     Criteria      Weight    PreferenceRelation    q
    ___________    ______    __________________    _

    'Criteria1'    4         'lesser-better'       0
    'Criteria2'    2         'lesser-better'       0
    'Criteria3'    3         'greater-better'      0

     Criteria      Weight     PreferenceRelation    q
    ___________    _______    __________________    _

    'Criteria1'    0.44444    'lesser-better'       0
    'Criteria2'    0.22222    'lesser-better'       0
    'Criteria3'    0.33333    'greater-better'      0

\end{verbatim} \color{black}
    \begin{par}
Concordance matrix C.
\end{par} \vspace{1em}
\begin{verbatim}
sizeC= sizeAG(1)^2;
C = cell(100);
C{100,100}=[]; % Memory allocation.
for i=1, sizeC;
    for j=1,sizeC;
        C{i,j}(1)=0;
        C{i,j}(2)=0;
        C{i,j}(3)=0;
        C{i,j}(4)=0;
    end
end

% Weight assessment.
% C{i,j}(1): K_p = 0, sum of the weights of the criteria where g(a)>g(b)+q;
%{
C{i,j}(2): K_e = 0,
sum of the weights of the criteria where -q<= g(a)- g(b)<= q;
%}
% C{i,j}(3): K_m = 0, sum of the weights of the criteria where g(a)<g(b)+q.
% C{i,j}(4): C(a,b), where "a" correpond to prefered alternative.

%
q=0; % Limit of indifference.
%
for i=1, sizeAG(1) % counter control for prefered alternative.
    for j=1,sizeAG(1) % counter control for alternative under test.
        for k=2, sizeAG(2) % selected column criteria.
            if i~=j
                if AG(i,k)>AG(j,k)+q
                    for m=1, sizeWW(1)
                        if WW(m)== ...
                        AG.Properties.VariableNames(k);
                            C{i,j}(1)=C{i,j}(1)+ WW(m,2); % concordance K element
                        end
                        %
                    end
                    % Calcular K
                end
                if -q <= AG(i,k)-AG(j,k)<=q
                    for m=1, sizeWW(1)
                        if WW(m)== ...
                        AG.Properties.VariableNames(k);
                            C{i,j}(2)=C{i,j}(2)+ WW(m,2);  % concordance K element
                        end
                        %
                    end
                    % Calcular K
                end
                if AG(i,k)<AG(j,k)+q
                   for m=1, sizeWW(1)
                        if WW(m)== ...
                        AG.Properties.VariableNames(k);
                            C{i,j}(3)= C{i,j}(3) + WW(m,2);  % concordance K element
                        end
                        %
                    end
                    % Calcular K
                end
            end  %i~=j
        end
    end
end

AG.Properties.VariableNames(2)
\end{verbatim}

        \color{lightgray} \begin{verbatim}
ans =

     4


ans =

     4


ans =

     4


ans = 

    'Criteria1'

\end{verbatim} \color{black}
    


\end{document}
    
